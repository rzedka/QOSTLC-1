\documentclass{article}
\usepackage[utf8]{inputenc}
\usepackage{cite}
\usepackage{amsmath,amssymb,amsfonts}
\usepackage{algorithmic}
\usepackage{graphicx}
\usepackage{import}
\usepackage{textcomp}
\usepackage{xcolor}
\usepackage{balance}
\usepackage{geometry}
\geometry{legalpaper,
portrait,
margin=2in,
lmargin = 2cm,
rmargin = 2cm,
}

\usepackage{multirow} % for tables with complicated header

\usepackage{booktabs,makecell,tabularx}
\renewcommand\theadfont{\small}
\renewcommand\theadgape{}

\def\BibTeX{{\rm B\kern-.05em{\sc i\kern-.025em b}\kern-.08em
    T\kern-.1667em\lower.7ex\hbox{E}\kern-.125emX}}
\begin{document}

\newcommand{\diag}{\operatorname{diag}}
\newcommand{\E}{\operatorname{E}} % expectation
\newcommand{\tr}{\operatorname{tr}} % trace
\newcommand{\iDFT}{\operatorname{IDFT}} % 
\newcommand{\DFT}{\operatorname{DFT}} % 
\newcommand{\iFFT}{\operatorname{IFFT}} % 
\newcommand{\iSFFT}{\operatorname{ISFFT}} % 
\newcommand{\SFFT}{\operatorname{SFFT}} % 
\newcommand{\FFT}{\operatorname{FFT}} % 
\newcommand{\suma}{\operatorname{sum}} % 

\title{Full-rate STLC for Four Receive Antennas - Notes }
\author{radim.zedka@vut.cz }
\date{March 2022}
\maketitle


% =================================================
\section{Problem Description}

Formula (6) in \cite{b_FullRate_STLC} is given by
\begin{equation} \label{eq_xi_orig}
    \xi = \rho \frac{\Big( \sum_{p=0}^{M-1} \gamma_{p} \pm 2\mathcal{R}\big\{ \epsilon_p\big\} \Big)^2}{4\sum_{p'=0}^{M-1}\gamma_{p'}},
\end{equation}
% \begin{equation} \label{eq_xi_orig}
%     \xi = \rho \frac{\frac{M^2}{M^2}\Big( \sum_{p=0}^{M-1} \gamma_{p} \pm 2\mathcal{R}\big\{ \epsilon_p\big\} \Big)^2}{4\sum_{p'=0}^{M-1}\gamma_{p'}} = \rho \frac{M^2\Big( \frac{1}{M}\sum_{p=0}^{M-1} \gamma_{p} \pm 2\frac{1}{M} \mathcal{R}\big\{ \epsilon_p\big\} \Big)^2}{4\sum_{p'=0}^{M-1}\gamma_{p'}}\ \overrightarrow{\text{for high }M}\ = \frac{\rho}{4} \sum_{p=0}^{M-1} \gamma_{p},
% \end{equation}
% \begin{equation} \label{eq_xi_orig}
%     \lim_{M\to\infty} \xi = \frac{\rho}{4} \sum_{p=0}^{M-1} \gamma_{p}
% \end{equation}
% \begin{equation} \label{eq_xi_orig}
%     \lim_{M\to\infty} \frac{1}{M} \sum_{p=0}^{M-1} \mathcal{R}\big\{ \epsilon_p\big\} = 0 
% \end{equation}
where 
\begin{equation} \label{eq_gamma_p}
    \gamma_p = \sum_{q=0}^{3} |h_{p,q}|^2,
\end{equation}
and
\begin{equation} \label{eq_epsilon_p}
    \epsilon_p = h_{p,0}h_{p,2}^* + h_{p,1}h_{p,3}^*.
\end{equation}
Each complex channel gain $h_{p,q}$ is composed of two i.i.d. normal variables $a_{p,q}, b_{p,q} \sim \mathcal{N}(0,1/2)$ which relate to channel gain by $h_{p,q} = a_{p,q} + jb_{p,q}$, where $j = \sqrt{-1}$.
Formula \eqref{eq_gamma_p} then evolves into
\begin{equation} \label{eq_gamma_p_2}
    \gamma_p = \sum_{q=0}^{3} |a_{p,q}|^2 + |b_{p,q}|^2,
\end{equation}
and $\mathcal{R}\big\{ \epsilon_p\big\}$ is expressed as
\begin{equation} \label{eq_epsilon_p_2}
    \mathcal{R}\big\{ \epsilon_p\big\} = a_{p,0}a_{p,2} + b_{p,0}b_{p,2} + a_{p,1}a_{p,3} + b_{p,1}b_{p,3}.
\end{equation}
Formula \eqref{eq_xi_orig} may be expanded into purely real-valued form
\begin{equation} \label{eq_xi_2}
    \xi = \frac{\rho}{4} \frac{\Big( \sum_{p=0}^{M-1} \sum_{q=0}^{3} |a_{p,q}|^2 + |b_{p,q}|^2 \pm 2\mathcal{R}\big\{ \epsilon_p\big\} \Big)^2}{ \sum_{p'=0}^{M-1} \sum_{q=0}^{3} |a_{p',q}|^2 + |b_{p',q}|^2}.
\end{equation}
After Lemma 1 in \cite{b_FullRate_STLC} the receiver SNR is calculated via
\begin{equation} \label{eq_xi_Lemma}
    \xi =  \frac{\rho}{4} \sum_{p=0}^{M-1} \sum_{q=0}^{3} |a_{p,q}|^2 + |b_{p,q}|^2 .
\end{equation}

% =================================================
\section{MATLAB Simulation}

In MATLAB I created a script which generates the normal-distributed variables $a_{p,q}$ and $b_{p,q}$ in vectors of $10^7$ samples each. This way I calculate the histogram of formula \eqref{eq_xi_2} and I attempt to approximate it with Gamma distribution with PDF given by
\begin{equation}\label{eq_Gamma_PDF}
 f_{\Xi}(\xi) =  \frac{\xi^{k/2 - 1} e^{-\frac{\xi}{2 \cdot c}}}{ (2c)^{k/2} \cdot \Gamma(k/2)} \quad \forall \quad \xi \in \langle 0, \infty).
\end{equation}
The parameters $k$ and $c$ are related to another pair of parameters - $n$, $m$ - (for greater convenience) according to $k = 2nM$ and $c = \rho/m$. Please note that for PDF approximations I set $\rho = 1$.
As the error metric of the PDF approximation I used mean-squared-error defined as
\begin{equation}\label{eq_PDF_error_metric}
 J = \int \big( f_{\Xi}(\xi)' - f_{\Xi}(\xi) \big)^2 d\xi,
\end{equation}
where $f_{\Xi}(\xi)'$ is the simulated data histogram and $f_{\Xi}(\xi)$ is the approximation.

Diversity gain of systems approximated by Gamma distribution is given by
\begin{equation}\label{eq_DivGain_limit}
\kappa = -\lim_{\rho \to \infty}\frac{d \log_{10}(\bar{\varepsilon}(\rho))}{d \log_{10}(\rho)} = k/2 = nM,
\end{equation}
where $\bar{\varepsilon}(\rho)$ is the average bit error rate at given $\rho$ and $n$ is the diversity gain coefficient. Table~\ref{table:1} then summarizes PDF approximation parameters $n$, $m$ for many values of $M$.


\begin{table}[t]
\caption{Diversity gain and PDF approximation parameters of \eqref{eq_xi_2} at given $M$}
\label{table:1}
\centering
\small
\begin{tabular} {|c|c|c|c|c|} 
\toprule
\thead {$M$} & \thead {$n$ } & \thead {$m$ } & \thead {$\kappa = nM$ } & \thead {$J$ [dB] }  \\
    \midrule
$1$ & $0.947$ & $1.528$ & $0.947$ & $-39.47$\\ 
$2$ & $0.879$ & $1.562$ & $1.658$ & $-44.92$\\ 
$4$ & $0.846$ & $1.595$ & $3.384$ & $-51.27$\\ 
$8$ & $0.828$ & $1.609$ & $6.624$ & $-54.65$\\ 
$16$ & $0.818$ & $1.613$ & $13.088$ & $-57.12$\\ 
$32$ & $0.812$ & $1.613$ & $25.984$ & $-59.82$\\ 
$64$ & $0.810$ & $1.612$ & $51.840$ & $-60.86$\\ 
$128$ & $0.809$ & $1.615$ & $103.55$ & $-61.15$\\ 
    \bottomrule
    \end{tabular}
\end{table}

In Figure.~\ref{fig_PDF_M1}, Figure.~\ref{fig_PDF_M2} and Figure.~\ref{fig_PDF_M128} I present the results of my MATLAB simulations for $M = 1$, $M = 2$ and $M = 128$. Formula \eqref{eq_xi_2} is approximated by \eqref{eq_Gamma_PDF} with $k = 2nM$ and $c = 1/m$ according to Table~\ref{table:1}.


\begin{figure}[!h] \centering
\includegraphics[scale=0.8]{images/Gamma_fit_M1.eps}
\caption{PDF of \eqref{eq_xi_2} approximated by \eqref{eq_Gamma_PDF} for $M=1$. }
\label{fig_PDF_M1}
\end{figure}

\begin{figure}[!h] \centering
\includegraphics[scale=0.8]{images/Gamma_fit_M2.eps}
\caption{PDF of \eqref{eq_xi_2} approximated by \eqref{eq_Gamma_PDF} for $M=2$ . }
\label{fig_PDF_M2}
\end{figure}

\begin{figure}[!h] \centering
\includegraphics[scale=0.8]{images/Gamma_fit_M128.eps}
\caption{PDF of \eqref{eq_xi_2} approximated by \eqref{eq_Gamma_PDF} for $M=128$ . }
\label{fig_PDF_M128}
\end{figure}


% \begin{figure}[!h]
% \begin{tabular}{ll}
% \includegraphics[scale=0.53]{images/PDF_M_32.eps}
% &
% \includegraphics[scale=0.53]{images/PDF_M_64.eps}
% \end{tabular}
% \caption{PDF of \eqref{eq_xi_2} and \eqref{eq_xi_Lemma} with scaled chi-squared PDFs \eqref{eq_ChiSquared_PDF_scaled} for $M=32$ and $M=64$. }
% \label{fig_PDF_1}
% \end{figure}

% \begin{figure}[!h]
% \begin{tabular}{ll}
% \includegraphics[scale=0.53]{images/PDF_M_256.eps}
% &
% \includegraphics[scale=0.53]{images/PDF_M_512.eps}
% \end{tabular}
% \caption{PDF of \eqref{eq_xi_2} and \eqref{eq_xi_Lemma} with scaled chi-squared PDFs \eqref{eq_ChiSquared_PDF_scaled} for $M=256$ and $M=512$. }
% \label{fig_PDF_2}
% \end{figure}

% =================================================
\section{Discussion}

My MATLAB simulation attepts show that \eqref{eq_xi_2} is close to Gamma distribution with $k=2nM$ and $c = \rho/m$. However, the diversity gain is not approaching $4M$, i.e. parameter $n$ is not approaching $4$, to the contrary, with rising number of antennas $M$ parameter $n$ approaches value $\sim 0.81$.   
I would be very grateful for your advice.


% ======================== References ====================
\begin{thebibliography}{00}

\bibitem{b_FullRate_STLC} S. -chan Lim and J. Joung, “Full-Rate Space–Time Line Code for Four Receive Antennas”, IEEE wireless communications letters, pp. 1-1, 2021.

\end{thebibliography}


\end{document}
