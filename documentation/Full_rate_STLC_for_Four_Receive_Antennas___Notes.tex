\documentclass{article}
\usepackage[utf8]{inputenc}
\usepackage{cite}
\usepackage{amsmath,amssymb,amsfonts}
\usepackage{algorithmic}
\usepackage{graphicx}
\usepackage{import}
\usepackage{textcomp}
\usepackage{xcolor}
\usepackage{balance}
\usepackage{geometry}
\geometry{legalpaper,
portrait,
margin=2in,
lmargin = 2cm,
rmargin = 2cm,
}

\usepackage{multirow} % for tables with complicated header

\def\BibTeX{{\rm B\kern-.05em{\sc i\kern-.025em b}\kern-.08em
    T\kern-.1667em\lower.7ex\hbox{E}\kern-.125emX}}
\begin{document}

\newcommand{\diag}{\operatorname{diag}}
\newcommand{\E}{\operatorname{E}} % expectation
\newcommand{\tr}{\operatorname{tr}} % trace
\newcommand{\iDFT}{\operatorname{IDFT}} % 
\newcommand{\DFT}{\operatorname{DFT}} % 
\newcommand{\iFFT}{\operatorname{IFFT}} % 
\newcommand{\iSFFT}{\operatorname{ISFFT}} % 
\newcommand{\SFFT}{\operatorname{SFFT}} % 
\newcommand{\FFT}{\operatorname{FFT}} % 
\newcommand{\suma}{\operatorname{sum}} % 

\title{Full-rate STLC for Four Receive Antennas - Notes }
\author{radim.zedka@vut.cz }
\date{March 2022}
\maketitle


% =================================================
\section{Problem Description}

Formula (6) in \cite{b_FullRate_STLC} is given by
\begin{equation} \label{eq_xi_orig}
    \xi = \rho \frac{\Big( \sum_{p=0}^{M-1} \gamma_{p} \pm 2\mathcal{R}\big\{ \epsilon_p\big\} \Big)^2}{4\sum_{p'=0}^{M-1}\gamma_{p'}},
\end{equation}
where 
\begin{equation} \label{eq_gamma_p}
    \gamma_p = \sum_{q=0}^{3} |h_{p,q}|^2.
\end{equation}
and
\begin{equation} \label{eq_epsilon_p}
    \epsilon_p = h_{p,0}h_{p,2}^* + h_{p,1}h_{p,3}^*.
\end{equation}
Each complex channel gain $h_{p,q}$ is composed of two i.i.d. normal variables $a_{p,q}, b_{p,q} \sim \mathcal{N}(0,1/2)$ which relate to channel gain by $h_{p,q} = a_{p,q} + jb_{p,q}$, where $j = \sqrt{-1}$.
Formula \eqref{eq_gamma_p} then evolves into
\begin{equation} \label{eq_gamma_p_2}
    \gamma_p = \sum_{q=0}^{3} |a_{p,q}|^2 + |b_{p,q}|^2,
\end{equation}
and $\mathcal{R}\big\{ \epsilon_p\big\}$ is expressed as
\begin{equation} \label{eq_epsilon_p_2}
    \mathcal{R}\big\{ \epsilon_p\big\} = a_{p,0}a_{p,2} + b_{p,0}b_{p,2} + a_{p,1}a_{p,3} + b_{p,1}b_{p,3}.
\end{equation}
Formula \eqref{eq_xi_orig} may be expanded into purely real-valued form
\begin{equation} \label{eq_xi_2}
    \xi = \frac{\rho}{4} \frac{\Big( \sum_{p=0}^{M-1} \sum_{q=0}^{3} |a_{p,q}|^2 + |b_{p,q}|^2 \pm 2\mathcal{R}\big\{ \epsilon_p\big\} \Big)^2}{ \sum_{p'=0}^{M-1} \sum_{q=0}^{3} |a_{p',q}|^2 + |b_{p',q}|^2}.
\end{equation}
After Lemma 1 in \cite{b_FullRate_STLC} the receiver SNR is calculated via
\begin{equation} \label{eq_xi_Lemma}
    \xi =  \frac{\rho}{4} \sum_{p=0}^{M-1} \sum_{q=0}^{3} |a_{p,q}|^2 + |b_{p,q}|^2 .
\end{equation}

% =================================================
\section{MATLAB Simulation}

In MATLAB I created a script which generates the normal-distributed variables $a_{p,q}$ and $b_{p,q}$ in vectors of $10^6$ samples each. This way I calculate the histogram of formulas \eqref{eq_xi_2} and \eqref{eq_xi_Lemma} and I attempt to approximate it with analytical formulas of probability density functions of the Chi-squared distribution with $k$ degrees of freedom
\begin{equation}\label{eq_ChiSquared_PDF}
f_{X}(\xi) = \frac{1}{2^{k/2} \cdot \Gamma(k/2)} \xi^{k/2 - 1} e^{-\frac{\xi}{2}} \quad \forall \quad \xi \in \langle 0, \infty),
\end{equation}
and since variance of the normal variables $a_{p,q}$ and $b_{p,q}$ equals $1/2$ the chi-squared variable must be multiplied by some positive real scaling factor $c$. The PDF of such a scaled chi-squared random variable is given by
\begin{equation}\label{eq_ChiSquared_PDF_scaled}
 f_{cX}(\xi) =  \frac{\xi^{k/2 - 1} e^{-\frac{\xi}{2 \cdot c}}}{ (2c)^{k/2} \cdot \Gamma(k/2)} \quad \forall \quad \xi \in \langle 0, \infty).
\end{equation}
In Figure.~\ref{fig_PDF_1} I present the results of my MATLAB simulations for $M = 256$ and  $M = 512$. Formula \eqref{eq_xi_2} is approximated by \eqref{eq_ChiSquared_PDF_scaled} with $k = 2M$ and $c = \rho/2$. Formula \eqref{eq_xi_Lemma} is approximated by 
\eqref{eq_ChiSquared_PDF_scaled} with $k = 8M$ and $c = \rho/8$. All simulations are done with $\rho = 1$.


\begin{figure}[!h]
\begin{tabular}{ll}
\includegraphics[scale=0.53]{images/PDF_M_32.eps}
&
\includegraphics[scale=0.53]{images/PDF_M_64.eps}
\end{tabular}
\caption{PDF of \eqref{eq_xi_2} and \eqref{eq_xi_Lemma} with scaled chi-squared PDFs \eqref{eq_ChiSquared_PDF_scaled} for $M=32$ and $M=64$. }
\label{fig_PDF_1}
\end{figure}

\begin{figure}[!h]
\begin{tabular}{ll}
\includegraphics[scale=0.53]{images/PDF_M_256.eps}
&
\includegraphics[scale=0.53]{images/PDF_M_512.eps}
\end{tabular}
\caption{PDF of \eqref{eq_xi_2} and \eqref{eq_xi_Lemma} with scaled chi-squared PDFs \eqref{eq_ChiSquared_PDF_scaled} for $M=256$ and $M=512$. }
\label{fig_PDF_2}
\end{figure}

% =================================================
\section{Discussion}

My MATLAB simulation attepts show that \eqref{eq_xi_2} is close to scaled chi-squared distribution with $k=2M$ and $c = \rho/2$. We can also see that even for $M$ rising from $32$ to $512$ the PDF of \eqref{eq_xi_2} is not approaching that of \eqref{eq_xi_Lemma}. 
I would be very grateful for your advice.


% ======================== References ====================
\begin{thebibliography}{00}

\bibitem{b_FullRate_STLC} S. -chan Lim and J. Joung, “Full-Rate Space–Time Line Code for Four Receive Antennas”, IEEE wireless communications letters, pp. 1-1, 2021.

\end{thebibliography}


\end{document}